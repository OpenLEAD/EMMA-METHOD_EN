\section{Control}

\subsection{Simulation}

In order to represent the coating application and simulate the mininal
environment variables that describe this process inside a hidreletric turbine,
it was created an simulation environment with help of the OpenRAVE - Open
Robotics and Animation Virtual Environment. The components of interest of the
turbine were modeled either using a Computer Aided Design software or were
aquisited using a laserscanner when the avaible information were not compatible
with the true aspect of the turbine (Blade).

MH12 imported 
coating gun is approximated by a cilinder with 300mm length and 50mm radius and
is attached to the tip of the endeffector.

%TODO determinar os eixos e ilustrar o ambiente de simulação

angle of attack of the turbine blades can vary from $0$ to $29^o$, however the
angle has to be chosen a priori beacuse the hidraulic circuit has to be active
to move the blades around its own axis and once dryed the blades get blocked.
It is going to be used $24^o$ as starting point.

the rotor can rotate freely, changing the position of the blades inside the
turbine, this task is hard to perform and must be avoided as much as possible

the coating gun must be at a distance of $235\pm 5$ mm during the operation

angle of incidence of the coating powder, i.e the angle between the coating gun
and the surface of the blade must obey a $90\pm 60^o$ tolerance, however it is
adviseable not surpass the $30^o$ mark.

Simulation of the corners

rotor angle from $0$ to $30$

$235\pm 5$ to the blade

angle of attack is $90^o$ with relaxation of $30^o$

y $-2970 \pm 250$ mm with 50mm steps (500mm band with 50mm step)

x $715 \pm 485$ mm with 50mm steps.

the width of the blade was segmented in 7 possibles positions


Result

found one position that it is possible to coat three of the corner, with
exception of the top right corner, with two of the variables fixed. $y=3220$ mm
and $x=1200mm$ and the rotor can be left at $0^o$ to make the process easier
without any harm to the coverage.

With these two variables fixed, the coverage of the entire blade is verified,
because the surface of the blade is not planar and its shape make the distance
to the robot vary and can violate one of the restrictions.

In order to try to accomplish full coverage and coat the right corner, some
relaxations were made in comparsion with the previous simulation:

angle of attack can vary from 0 to 29

the rotor can be rotated from 0 to 30 with 3 step

the tolerance of the angle of incidence is increased to 60

z is from -1240 to 1240 due mechanical restrictions between the rail and the
turbine chamber/arc

The height adjustment of the robot base was prefered against the rotation of the
rotor or the blade, because the former needs a logistic effort to disassemble
and reassemble the whole system in order to ensure the safety of operation and
the change of angle of attack has the hard limitation of the hydraulic blocking.


\subsection{Trajectory Planning}

paralals and meridians

paralel is a 4th degree polinom

\cite{juttler2002least} uses the normal to better describe the analitical
function that represent a point cloud, and not only the points.

The rotor is situaded inside a sphere and this is explored as the paralels are
numerically calculated as the intersection of the surface of a growing sphere
and the surface of the blade previously calculated.

 3 mms offset
between paralels


 