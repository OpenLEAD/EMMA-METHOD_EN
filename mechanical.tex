\section{Mechanical Solution}

Due the weight of the manipulator and HSE regulations, all the movementation of
heavy loads inside the turbine cannot be performed by human effort. Therefore a
mechanical base must be able of transport the robot from the hatch until the
coating position. 

the sizing must take into account the weigth of the system, the dynamics loads
of operation of the robot anf the coating gun thrust.

\subsection{Concept}

The final concept is a mechanical base, with two independent rails, primary and
secodary, and consists of four joints, with a P-R-P-P or
Prismatic-Rotational-Prismatic-Prismatic arrengement. The especific purpose of each joint is explained as follow:

\begin{itemize}
  \item \textbf{Prismatic 1:} The first joint consist of the so called primary
  rail and is a set of modules that allows the robot to movement paralel to the
  turbine axis, from the hatch to the region between the blades and the
  distributor.
  \item \textbf{Rotational:} The rotational joint connects the primary to the
  secondary rail and make possible the alignment of the robot with the blade to
  be coated.
  \item \textbf{Prismatic 2:} The second prismatic joint is the so called
  secondary rail and when aligned with the blade allow the robot to movement
  through its extension.
  \item \textbf{Prismatic 3:} The last joint consists by a scissor jack device,
  with stroke of $200mm$.
\end{itemize}

In this concept, the manipulator is movimented through the primary rail to the
region neat to the blade. Then, the secundary rail is attached to the structure
with the base of the robot. The orientation of the secundary rail is defined by
the rotacional joint. 

In order to cancel the moments generated at the contact point between the rail
and the base of the robot, the setup consists in two rail and four carriages. In
this scenario the moments in the orthogonal direction to the rail are cancelled
and the loads are divided in more components.

The figure \ref{fig::mec::module} show one of the modules of the primary rail.
The module can be repeated alongside the turbine's longitudinal axis and forming
the complete structure. The only component that vary according to the section of
the turbine is the support feet and ancor ``arms" of the module, all the other
components have the same measurements. The assembly must be practical and fast, 
as the disassembly, and must allow the correction of errors in the initial
positioning of the structure.

In order to cancel the moments at the contact point between the rail and the
carriage, the setup consists in two rail and four carriages. In this scenario
the reaction moments are cancelled by a force couple provided by each couple of
carriages and the loads are divided in more components.

\subsubsection{Ancor}

Due the shape of turbine inside and how slippery is its surface, the anchorage
is fundamental to maintain the structure in place and also to assure the
rigidity of the system. The base must be able to sustain all the forces involded in the coating process without any
elastic deformation or considerable vibrations. The fixation can be made via
magnetic attachment and several magnetic bases are going to be placed to ensure
the redundancy of the base. 

The magnetic bases are non permanent fixation elements that can have the
magnetic field direction changed and, therefore, its attraction power activated
or desactivated.

In order to change the angle of the rotor, some sections of the primary rail
must be disassembled.

\subsection{Sizing}

%TODO tensions?
The tensions generated by the forces and momentum during operation of the
manipulator and also the elastic deformation inflicted in the base were modeled
using Finite Elements. Unidimensional elements of $25mm$ and quadratic order
were used and the beam properties, using the aluminium profile as reference,
were considered as follow:

%TODO tabela de propriedades ??

The forces and torques applied during the simulation were representing the worst
case scenario and for that purpose were choosen the emergency brake values from
the model.

$0,47mm$ translation of the base, and a angular variation of $0,00149^o$ at the
base with a $0,522mm$ offset at $2m$.
